\documentclass[a4paper,zihao=-4,UTF8]{ctexart}
\pagestyle{plain}
\date{}
\usepackage[top=1.0in, bottom=1.0in, left=1.25in, right=1.25in]{geometry} 
\setlength{\baselineskip}{20pt}
\usepackage{titlesec}
\usepackage{zhnumber}
\usepackage{abstract}
\usepackage{indentfirst}
\setlength{\parindent}{2em}
\usepackage{color}   % May be necessary if you want to color links
\usepackage{hyperref}
\hypersetup{
    colorlinks=true, %set true if you want colored links
    linktoc=all,     %set to all if you want both sections and subsections linked
    linkcolor=black,  %choose some color if you want links to stand out
}
\usepackage{booktabs}
\usepackage{multirow}
\usepackage{graphicx}
\usepackage{enumitem}
\usepackage{amsmath,amssymb}
\usepackage{algorithm}
\usepackage{algpseudocode}

% numbering restarts at each section
\numberwithin{figure}{section}  
\numberwithin{table}{section}
\numberwithin{equation}{section}
\numberwithin{algorithm}{section}

\usepackage{xcolor,colortbl}
\usepackage[capitalize]{cleveref}  % Support for easy cross-referencing
\crefname{figure}{图}{图}
\crefname{table}{表}{表}
\crefname{equation}{公式}{公式}
\crefname{algorithm}{算法}{算法}


% set font for English
\usepackage{fontspec}
\setmainfont{Times New Roman}
\setsansfont{Arial}
\setromanfont{Times New Roman}
% the following two are to support combined Chinese and English
\newcommand{\hei}[1]{\heiti\sffamily #1}
\newcommand{\song}[1]{\songti\rmfamily #1}


% setup style for table of contents
\usepackage{titlesec}
\usepackage{titletoc}
\titlecontents{section}[0em]
  {\zihao{-4}\bfseries\heiti}
  {\contentspush{第\zhnumber{\thecontentslabel}章\quad}}
  {\hspace{0em}}
  {\titlerule*[2pt]{$.$}\contentspage}
\titlecontents{subsection}[2em]{\zihao{-4}}{\contentslabel{2em}}{\hspace{-2em}}{\titlerule*[0.2pc]{$.$}\contentspage}
\titlecontents{subsubsection}[4em]{\zihao{-4}}{\contentslabel{3em}}{\hspace{-3em}}{\titlerule*[0.2pc]{$.$}\contentspage}
\titlecontents{title}{\centering\zihao{2}\heiti 目录}

% update references to Chinese style, font size and type can also be set here
\usepackage{caption}
\titleformat{\section}{\centering\hei\zihao{-2}}{第\chinese{section}章}{1em}{}[]
\titleformat{\subsection}{\hei\zihao{-3}}{\arabic{section}.\arabic{subsection}}{1em}{}[]
\titleformat{\subsubsection}{\hei\zihao{4}}{\arabic{section}.\arabic{subsection}.\arabic{subsubsection}}{1em}{}[]
\renewcommand{\abstractnamefont}{\hei\zihao{-2}}
\renewcommand{\abstracttextfont}{}
% \renewenvironment{abstract}{\begin{center}{\hei\zihao{-2}摘~要}\end{center}\par} \\
% \newenvironment{abstract-en}{\begin{center}{\hei\zihao{-2}Abstract}\end{center}\par}

\renewcommand{\figurename}{图}
\renewcommand{\tablename}{表}
\renewcommand{\thefigure} {\arabic{section}-\arabic{figure}}
\renewcommand{\theequation}{\arabic{section}.\arabic{equation}}
\renewcommand{\thetable} {\arabic{section}-\arabic{table}}
\captionsetup{labelsep=space} 


% title text
\newcommand{\mytitle}{番茄炒蛋}

% header
\usepackage{fancyhdr}
\pagestyle{fancy}
\fancyhead[L]{\song{\zihao{-5}\mytitle}}
\fancyhead[R]{\song{\zihao{-5}\leftmark}}
\renewcommand\headrulewidth{.5pt}
\renewcommand\footrulewidth{0pt}

% for hype-reference of "section*" to work properly
\usepackage{xparse}
\let\oldsection\section
\makeatletter
\newcounter{@secnumdepth}
\RenewDocumentCommand{\section}{s o m}{%
  \IfBooleanTF{#1}
    {\setcounter{@secnumdepth}{\value{secnumdepth}}% Store secnumdepth
     \setcounter{secnumdepth}{0}% Print only up to \chapter numbers
     \oldsection{#3}% \section*
     \setcounter{secnumdepth}{\value{@secnumdepth}}}% Restore secnumdepth
    {\IfValueTF{#2}% \section
       {\oldsection[#2]{#3}}% \section[.]{..}
       {\oldsection{#3}}}% \section{..}
}
\makeatother

\makeatletter
\renewcommand{\ALG@name}{算法}
\renewcommand\thealgorithm{\arabic{section}-\arabic{algorithm}} 
\makeatother

% use GB/T 7714—2015 BibTeX Style
\usepackage{gbt7714}
\bibliographystyle{gbt7714-numerical}